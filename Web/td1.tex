\documentclass[10pt]{article}
\usepackage[francais]{babel}
\usepackage[T1]{fontenc}
\usepackage{listings}
\usepackage[utf8]{inputenc}
\usepackage{multicol}
%%configuration de listings
\lstset{
language=XML,
basicstyle=\ttfamily\small, %
%identifierstyle=\color{red}, %
keywordstyle=\color{blue}, %
%stringstyle=\color{black!60}, %
commentstyle=\it\color{green!95!yellow!1}, %
columns=flexible, %
tabsize=2, %
extendedchars=true, %
showspaces=false, %
showstringspaces=false, %
numbers=left, %
numberstyle=\tiny, %
breaklines=true, %
breakautoindent=true, %
captionpos=b
}

\usepackage{xcolor}

\definecolor{Zgris}{rgb}{0.87,0.85,0.85}

\newsavebox{\BBbox}
\newenvironment{DDbox}[1]{
\begin{lrbox}{\BBbox}\begin{minipage}{\linewidth}}
{\end{minipage}\end{lrbox}\noindent\colorbox{white}{\usebox{\BBbox}} \\
[.5cm]}
\title{HLIN103 - Presentation des données du Web\\TD1}
\author{Meryll \textsc{Essig} & Tamara \textsc{Rocacher}}
\begin{document}
\maketitle

\section{Question 1:}
\begin{multicols}{2}
  \subsection{DTD1 Arbre 1}


  \begin{DDbox}{0.4}
  \begin{lstlisting}

  <presse>
    <journal>
      <article titre="Hey" auteur="01">
        <corps>
  	     Texte de l'article.
        </corps>
      </article>
    </journal>
    <journalistes>
      <journaliste idJ="01">
        <nom>Michel</nom>
        <prenom>Michel</prenom>
      </journaliste>
    </journalistes>
  </presse>

  \end{lstlisting}
  \end{DDbox}

  \subsection{DTD1 Arbre 2}



  \begin{DDbox}{0.4}
  \begin{lstlisting}

  <presse>
    <journal>
      <article titre="lel" auteur="06">
        <corps>
  	     Texte de l'article.
        </corps>
      </article>
    </journal>
    <journalistes>
      <journaliste idJ="06">
        <anonyme />
      </journaliste>
      <journaliste>
        <pseudonyme>Michou</pseudonyme>
      </journaliste>
    </journalistes>
  </presse>

  \end{lstlisting}
  \end{DDbox}
\end{multicols}
\newpage
\begin{multicols}{2}
  \subsection{DTD2 Arbre 1}



  \begin{DDbox}{\linewidth}
  \begin{lstlisting}

  <batiment>
    <etage>
      <description>
        Etage du boss.
      </description>
      <bureau>
        <code>0160</code>
        <personne>CEO</personne>
      </bureau>
    </etage>
  </batiment>


  \end{lstlisting}
  \end{DDbox}
\subsection{DTD2 Arbre 2}



\begin{DDbox}{\linewidth}
\begin{lstlisting}

  <batiment>
    <etage>
      <description>
        Etage du personnel.
      </description>
      <bureau>
        <code>002</code>
      </bureau>
      <salle>
        <nombrePlaces>10</nombrePlaces>
      </salle>
      <salle>
        <nombrePlaces>15</nombrePlaces>
      </salle>
    </etage>
  </batiment>

  \end{lstlisting}
  \end{DDbox}
\end{multicols}
  \section{QUESTION 2 :}

\begin{multicols}{2}

\subsection{XML 1 :}


    \begin{DDbox}{\linewidth}
    \begin{lstlisting}

  <!DOCTYPE C [
  <!ELEMENT C (A|B|C)*>
  <!ELEMENT A EMPTY>
  <!ELEMENT B EMPTY>
  ]>

  <!DOCTYPE C [
  <!ELEMENT C (B | (B, A) |C)*>
  <!ELEMENT A EMPTY>
  <!ELEMENT B EMPTY>
  ]>

  \end{lstlisting}
  \end{DDbox}

  \subsection{XML 2 :}


    \begin{DDbox}{\linewidth}
    \begin{lstlisting}

  <!DOCTYPE C [
  <!ELEMENT C (A|B|C|D)*>
  <!ELEMENT A EMPTY>
  <!ELEMENT B EMPTY>
  <!ELEMENT D EMPTY>
  ]>

  <!DOCTYPE C [
  <!ELEMENT C (B|C|D| (C,A) | (D,A) )*>
  <!ELEMENT A EMPTY>
  <!ELEMENT B EMPTY>
  <!ELEMENT D EMPTY>
  ]>
\end{lstlisting}
\end{DDbox}

\end{multicols}
  \begin{multicols}{2}

\subsection{XML 3 :}

\begin{DDbox}{\linewidth}
\begin{lstlisting}

<!DOCTYPE C [
<!ELEMENT C (B*, C*, B)>
]>

<!DOCTYPE C [
<!ELEMENT C (B*| C*)*>
]>
\end{lstlisting}
\end{DDbox}


\subsection{XML 4 :}

\begin{DDbox}{\linewidth}
\begin{lstlisting}

<!DOCTYPE C [
<!ELEMENT C (B*,C*|A|D)>
<!ELEMENT B (A|D)>
]>

<!DOCTYPE C [
<!ELEMENT C (B|C)*|(A|D)>
<!ELEMENT B (A|D)*>
]>
\end{lstlisting}
\end{DDbox}


\end{multicols}
  \begin{multicols}{2}

  \subsection{XML 5 :}

  \begin{DDbox}{\linewidth}
  \begin{lstlisting}

  <!DOCTYPE C [
  <!ELEMENT C EMPTY>
  ]>

  <!DOCTYPE C [
  <!ELEMENT C (B*)>
  ]>
\end{lstlisting}
\end{DDbox}

\subsection{XML 6 :}

\begin{DDbox}{\linewidth}
\begin{lstlisting}

  <!DOCTYPE C [
  <!ELEMENT C B*>
  <!ELEMENT B #PCDATA>
  <!ATTLIST B id ID #REQUIRED>
  <!ATTLIST B friend #IMPLIED>
  ]>

  <!DOCTYPE C [
  <!ELEMENT C B+>
  <!ELEMENT B #PCDATA>
  <!ATTLIST B id #IMPLIED>
  <!ATTLIST B friend #REQUIRED>
  ]>
\end{lstlisting}
\end{DDbox}
\end{multicols}
  \section{QUESTION 3 :}





\begin{DDbox}{\linewidth}
\begin{lstlisting}
<!DOCTYPE Tweet [
<!ELEMENT Tweet (Corps, Loc, Syst, UrlCont*)>
<!ATTLIST Tweet idT ID #REQUIRED>
<!ATTLIST Tweet userId IDREF #REQUIRED>
<!ATTLIST Tweet userProfil CDATA #REQUIRED>
<!ATTLIST Tweet userName PCDATA #REQUIRED>
<!ATTLIST Tweet date CDATA #REQUIRED>
<!ATTLIST Tweet fuseau CDATA #REQUIRED>
<!ATTLIST Tweet rt CDATA #IMPLIED>
<!ATTLIST Tweet idRep IDREF #IMPLIED>


<!ELEMENT  Corps (#PCDATA | Hashtag | URef)*>
<!ATTLIST Corps langue PCDATA #REQUIRED>
<!ATTLIST Corps taille PCDATA #REQUIRED>
<!ATTLIST Corps type PCDATA #REQUIRED>
<!ATTLIST Corps couleur PCDATA #REQUIRED>
<!ELEMENT Hashtag #CDATA>
<!ELEMENT URef #CDATA>

<!ELEMENT Loc (Coord, Ville, Pays)>
<!ELEMENT  Coord EMPTY>
<!ATTLIST Coord Lat CDATA #REQUIRED>
<!ATTLIST Coord Long CDATA #IMPLIED>
<!ELEMENT Ville #PCDATA>
<!ELEMENT Pays #PCDATA>

<!ELEMENT Syst #PCDATA>
<!ATTLIST Syst Browser PCDATA #IMPLIED>

<!ELEMENT UrlCont #PCDATA>
]>

<!DOCTYPE User [
<!ELEMENT User (Name, Descript, Pict)>
<!ATTLIST User Url CDATA #REQUIRED>
<!ATTLIST User IdU ID #REQUIRED
<!ATTLIST User abo PCDATA #REQUIRED>
<!ATTLIST User suiv PCDATA #REQUIRED>

<!ELEMENT Name #PCDATA>
<!ELEMENT Descript #PCDATA>
<!ELEMENT Pict #CDATA)>
<!ATTLIST Pict taille PCDATA #REQUIRED>
]>
\end{lstlisting}
\end{DDbox}

\section{QUESTION 5 :}
\subsection{DTD 1:}
L'élément C est décrit deux fois, la DTD est donc invalide.
\subsection{DTD 2:}
\begin{DDbox}{\linewidth}
\begin{lstlisting}
<EMPTY />
\end{lstlisting}
\end{DDbox}
La DTD ne peut pas être vide, elle a donc un premier élément nommé EMPTY, qui est vide.
\subsection{DTD 3:}
Cette DTD donne lieu à un arbre récursif, donc infini, car l'élément A appel l'élément B et l'élément B rappel A.\\
\begin{DDbox}{\linewidth}
\begin{lstlisting}
<A>
  <B>
    <A>
      <B>
      ...
      </B>
    </A>
  </B>
</A>
\end{lstlisting}
\end{DDbox}
Il faudrait décrire un des deux élément avec une fin, donc possiblement sans sous-élément:\\
\begin{DDbox}{\linewidth}
\begin{lstlisting}

<! ELEMENT A (B*) >
ou
<! ELEMENT B (A*) >

\end{lstlisting}
\end{DDbox}

\subsection{DTD 4:}
\begin{DDbox}{\linewidth}
\begin{lstlisting}
<C>
  <C><C/></C>
  <C/>
</C>
\end{lstlisting}
\end{DDbox}

\subsection{DTD 5:}
L'élément C n'est pas décrit correctement. Soit il est vide, soit il contient un certains nombre de fils C:\\
\begin{DDbox}{\linewidth}
\begin{lstlisting}

(EMPTY | B*), qui peut se simplfier par (B*)
\end{lstlisting}
\end{DDbox}

\subsection{DTD 6:}
\begin{DDbox}{\linewidth}
\begin{lstlisting}
<C>
  <B/>
  <B/>
  <C/>
  <C/>
  <C/>
  <C/>
</C>
\end{lstlisting}
\end{DDbox}
\end{document}
